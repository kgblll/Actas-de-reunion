\documentclass[a4paper,12pt]{letter}
\usepackage[spanish]{babel}
\usepackage[utf8]{inputenc}
\usepackage{geometry}
\usepackage{url}
\usepackage{calc}
\usepackage{setspace}
\usepackage{fixltx2e}
\usepackage{graphicx}
\usepackage{multicol}
\usepackage[normalem]{ulem}
%% Please revise the following command, if your babel
%% package does not support es-ES
\usepackage{color}
\usepackage{hyperref}

\pagestyle{empty}
%\makelabels
\topmargin -1.0 cm
\textheight 20.5 cm
\geometry{top=2cm, left=3cm, right=3cm, bottom=3cm}
 
\begin{document}


\begin{letter}
{}

\fbox{
    \parbox{15cm}{
        \setlength{\baselineskip}{18pt}
        \textbf{\textcolor[rgb]{0.000,0.502,0.502}{{\LARGE Acta de reunión de 
                Dr.Scratch:}}}\\
        \setlength{\baselineskip}{0pt}
        \textbf{\textcolor[rgb]{0.000,0.502,0.502}{{\LARGE Fomentando la 
                    creatividad y vocaciones científicas con Scratch}}}\\    
    }
}\\

\vspace{1cm}
\setlength{\parskip}{0.5mm}
\parbox{70mm}{
    \textbf{
        \textcolor[rgb]{0.000,0.502,0.502}{\large Asistentes:}
        \begin{itemize}
            \setlength{\parskip}{0.5mm}
            \item{\textcolor[rgb]{0.502,0.502,0.502}{Gregorio Robles}}
            \item{\textcolor[rgb]{0.502,0.502,0.502}{Jesús Moreno}}
            \item{\textcolor[rgb]{0.502,0.502,0.502}{Eva Hu}}
            \item{\textcolor[rgb]{0.502,0.502,0.502}{Mari Luz Aguado}}
        \end{itemize}
    }
}
\hfill
\parbox{90mm}{
    \bf{
        \textcolor[rgb]{0.000,0.502,0.502}{\large Entidad:}
        \textcolor[rgb]{0.502,0.502,0.502}{Universidad Rey Juan Carlos}\\ 
        \textcolor[rgb]{0.000,0.502,0.502}{\large Fecha:}
        \textcolor[rgb]{0.502,0.502,0.502}{28 de mayo de 2015} \\
        \textcolor[rgb]{0.000,0.502,0.502}{\large Lugar:}\\
        \textcolor[rgb]{0.502,0.502,0.502}{Camino del Molino s/n \\
                                           28943 Fuenlabrada (Madrid)} \\
      
  
   }
}



\vspace{0.75cm}
\textbf{{\LARGE Temas tratados}}
\vspace{0.5cm}


En la reunión realizada el \textbf {28 de mayo de 2015} en el {\bf Campus de
Fuenlabrada de la Universidad Rey Juan Carlos} y dirigida por Dr.Gregorio Robles
se trataron los siguientes temas:\\

\begin{enumerate}
    
    \item{\textbf {Desarrollo de la plataforma de Dr.Scratch}}
    \begin{itemize}
            \item {Trabajando en la creación de un espacio para las organizaciones.}
            \item {Casi solucionado el problema encontrado con los modales para 
                    incluir la informaición de cómo mejorar y avanzar.}
            \item {Se ha ampliado la información guardada en la base de datos
                    de los ficheros analizados. Ahora se guardan todas las puntuaciones
                    obtenidas y el idioma desde el que se ha accedido.}
    \end{itemize}
    
    \item{\textbf {Organización para el desarrollo.}}
    \begin{itemize}
            \item {Hemos creado una rama más en Github para utilizarla a la hora
                    de desarrollar nuevo código, ya que ahora hay mucho código 
                    nuevo que integrar.}
    \end{itemize}

    \item{\textbf {Migración a Azure:}}
    \begin{itemize}
        \item {Está todo el código de preproducción migrado a Azure.}
        \item {Hay que resolver el problema encontrado en Azure con la instalación de Mysql.}
        \item {Se plantea probar producción en Azure una vez instalemos Mysql.}
    \end{itemize}

	\item{\textbf {Traducción de la plataforma a lenguas cooficiales.}}
    \begin{itemize}
        \item {Ya tenemos generados los ficheros .po para la  traducción y se van
                a enviar a voluntarios que se han ofrecido ha realizazla.}
           \item {Además hemos generado ficheros de traducción también para el
                portugués y el húngaro, ya que hay dos colaboradores 
                dispuestos a realizar tal traducción.}
    \end{itemize}

\end{enumerate}

\end{letter}

\end{document}
