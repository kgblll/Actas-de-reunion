\documentclass[a4paper,12pt]{letter}
\usepackage[spanish]{babel}
\usepackage[utf8]{inputenc}
\usepackage{geometry}
\usepackage{url}
\usepackage{calc}
\usepackage{setspace}
\usepackage{fixltx2e}
\usepackage{graphicx}
\usepackage{multicol}
\usepackage[normalem]{ulem}
%% Please revise the following command, if your babel
%% package does not support es-ES
\usepackage{color}
\usepackage{hyperref}

\pagestyle{empty}
%\makelabels
\topmargin -1.0 cm
\textheight 20.5 cm
\geometry{top=2cm, left=3cm, right=3cm, bottom=3cm}
 
\begin{document}


\begin{letter}
{}

\fbox{
    \parbox{15cm}{
        \setlength{\baselineskip}{18pt}
        \textbf{\textcolor[rgb]{0.000,0.502,0.502}{{\LARGE Acta de reunión de 
                Dr.Scratch:}}}\\
        \setlength{\baselineskip}{0pt}
        \textbf{\textcolor[rgb]{0.000,0.502,0.502}{{\LARGE Fomentando la 
                    creatividad y vocaciones científicas con Scratch}}}\\    
    }
}\\

\vspace{1cm}
\setlength{\parskip}{0.5mm}
\parbox{70mm}{
    \textbf{
        \textcolor[rgb]{0.000,0.502,0.502}{\large Asistentes:}
        \begin{itemize}
            \setlength{\parskip}{0.5mm}
            \item{\textcolor[rgb]{0.502,0.502,0.502}{Gregorio Robles}}
            \item{\textcolor[rgb]{0.502,0.502,0.502}{Jesús Moreno}}
            \item{\textcolor[rgb]{0.502,0.502,0.502}{Eva Hu}}
            \item{\textcolor[rgb]{0.502,0.502,0.502}{Mari Luz Aguado}}
        \end{itemize}
    }
}
\hfill
\parbox{90mm}{
    \bf{
        \textcolor[rgb]{0.000,0.502,0.502}{\large Entidad:}
        \textcolor[rgb]{0.502,0.502,0.502}{Universidad Rey Juan Carlos}\\ 
        \textcolor[rgb]{0.000,0.502,0.502}{\large Fecha:}
        \textcolor[rgb]{0.502,0.502,0.502}{15 de enero de 2015} \\
        \textcolor[rgb]{0.000,0.502,0.502}{\large Lugar:}\\
        \textcolor[rgb]{0.502,0.502,0.502}{Camino del Molino s/n \\
                                           28943 Fuenlabrada (Madrid)} \\
      
  
   }
}

\vspace{1cm}
\textbf{{\LARGE Temas tratados}}

\vspace{0.5cm}


En la reunión realizada el \textbf {15 de enero de 2015} en el {\bf Campus de
Fuenlabrada de la Universidad Rey Juan Carlos} y dirigida por Dr.Gregorio Robles
se trataron los siguientes temas:\\

\begin{enumerate}
    
    \item {\textbf {Comentar las últimas modificaciones realizadas en Dr.Scratch:}}
    \begin{itemize}
        \item {Cambio de la apariencia de la página principal, de los colores
                iniciales que eran en azul a tonos próximos al rojo para que
                sean coherentes con el logo.}
        \item {Añadido un log que nos permita saber qué proyectos han sido 
                analizados.}
    \end{itemize}

    \item{\textbf {Contacto con usuarios de Dr.Scratch:}}
    \begin{itemize}
        \item {Cait Sydney, que trabaja en Google en el proyecto CSFirst quiere
                analizar un número de proyectos muy alto. Tiene 10.000 alumnos,
                de los cuales, cada uno tiene 5 proyectos.}
        \item {Observar el comportamiento del servidor cuando empiece a analizar
                proyectos, para ver si lo soporta.}
        \item {Cait nos sugiere la posibilidad de añadir sugerencias según la 
                puntuación obtenida en las distintas habilidades de 
                programación, idea en la que ya estamos trabajando.}
        \item {Computing at School nos comenta que no se pueden analizar
                proyectos de la versión 1.4 de Scratch.}

    \end{itemize}

	\item{\textbf {Planteamiento de nuevas modificaciones en la plataforma de 
                    Dr.Scratch:}}
    \begin{itemize}
        \item {Introducción de análisis por url.}
        \item {Introducción en los dashboards de algún mecanismos que ayude al 
                usuario a mejorar sus habilidades de programación.}
        \item {Posibilidad de análisis de proyectos 1.4 de Scratch.}
        \item {Hay que pensar cómo hacer el registro de usuarios(menores de edad).} 
    \end{itemize}
  
    \vspace{0.5cm}
	\item {\textbf {Coordinación de todo el grupo mediante github:}}
    \begin{itemize}
        \item {Nos hemos puesto de acuerdo en tener dos ramas en github: master
                y development para ir sincronizándolas con las máquinas de 
                preproducción y producción.}
    \end{itemize}

    \item {\textbf {Twitter y blog:}}
        \begin{itemize}
            \item {Cuenta de Twitter funcionando muy bien.}
            \item {Habrá que crear un blog más adelante.}
        \end{itemize}
\end{enumerate}


\vspace{1cm} 
\textbf{\LARGE Revisión de números alcanzados}

\setlength{\parskip}{4mm}
\begin{enumerate}
    \item {\textbf{Plataforma web (y aplicación móvil) de Dr.Scratch:}}
	
    \setlength{\parskip}{2mm}
    \begin{itemize}
      	\setlength{\parskip}{0mm}        
      
	    \item {Proyectos analizados: 588 proyectos.}
	    \item {Número de lenguas de la plataforma: 1 lengua (inglés).}
    \end{itemize}

    \item {\textbf{Twitter y blogs en diciembre:}}

    \setlength{\parskip}{2mm}
    \begin{itemize}
      	\setlength{\parskip}{0mm}        
       	\item {Número de tweets: 16.}  
	    \item {Número de followers: 233.}
	    \item {Número de menciones en Twitter: 37.}
        \item {Visitas al perfil de Twitter en diciembre: 1067.}
    \end{itemize}

\end{enumerate}

\end{letter}

\end{document}
