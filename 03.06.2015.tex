\documentclass[a4paper,12pt]{letter}
\usepackage[spanish]{babel}
\usepackage[utf8]{inputenc}
\usepackage{geometry}
\usepackage{url}
\usepackage{calc}
\usepackage{setspace}
\usepackage{fixltx2e}
\usepackage{graphicx}
\usepackage{multicol}
\usepackage[normalem]{ulem}
%% Please revise the following command, if your babel
%% package does not support es-ES
\usepackage{color}
\usepackage{hyperref}

\pagestyle{empty}
%\makelabels
\topmargin -1.0 cm
\textheight 20.5 cm
\geometry{top=2cm, left=3cm, right=3cm, bottom=3cm}
 
\begin{document}


\begin{letter}
{}

\fbox{
    \parbox{15cm}{
        \setlength{\baselineskip}{18pt}
        \textbf{\textcolor[rgb]{0.000,0.502,0.502}{{\LARGE Acta de reunión de 
                Dr.Scratch:}}}\\
        \setlength{\baselineskip}{0pt}
        \textbf{\textcolor[rgb]{0.000,0.502,0.502}{{\LARGE Fomentando la 
                    creatividad y vocaciones científicas con Scratch}}}\\    
    }
}\\

\vspace{1cm}
\setlength{\parskip}{0.5mm}
\parbox{70mm}{
    \textbf{
        \textcolor[rgb]{0.000,0.502,0.502}{\large Asistentes:}
        \begin{itemize}
            \setlength{\parskip}{0.5mm}
            \item{\textcolor[rgb]{0.502,0.502,0.502}{Gregorio Robles}}
            \item{\textcolor[rgb]{0.502,0.502,0.502}{Jesús Moreno}}
            \item{\textcolor[rgb]{0.502,0.502,0.502}{Eva Hu}}
            \item{\textcolor[rgb]{0.502,0.502,0.502}{Mari Luz Aguado}}
        \end{itemize}
    }
}
\hfill
\parbox{90mm}{
    \bf{
        \textcolor[rgb]{0.000,0.502,0.502}{\large Entidad:}
        \textcolor[rgb]{0.502,0.502,0.502}{Universidad Rey Juan Carlos}\\ 
        \textcolor[rgb]{0.000,0.502,0.502}{\large Fecha:}
        \textcolor[rgb]{0.502,0.502,0.502}{3 de junio de 2015} \\
        \textcolor[rgb]{0.000,0.502,0.502}{\large Lugar:}\\
        \textcolor[rgb]{0.502,0.502,0.502}{Camino del Molino s/n \\
                                           28943 Fuenlabrada (Madrid)} \\
      
  
   }
}



\vspace{0.75cm}
\textbf{{\LARGE Temas tratados}}
\vspace{0.5cm}


En la reunión realizada el \textbf {3 de junio de 2015} en el {\bf Campus de
Fuenlabrada de la Universidad Rey Juan Carlos} y dirigida por Dr.Gregorio Robles
se trataron los siguientes temas:\\

\begin{enumerate}
    
    \item{\textbf {Script para reiniciar getsb2:}}
    \begin{itemize}
            \item {Estamos teniendo numerosos problemas con la funcionalidad 
                    de análisis de proyectos mediante url que realizamos a 
                    través de getsb2.}
            \item {Ahora mismo es necesario reiniciar constantemente el servidor
                    que utilizamos para dicha funcionalidad.}
            \item {Vamos a realizar un script que se encargue de reiniciar dicho
                    servidor cuando deje de funcionar.}
    \end{itemize}
    
    \item{\textbf {Estudio de getsb2:}}
    \begin{itemize}
            \item {Deducimos que con el script se resolverán la mayor parte
                    de los problemas, pero es necesario que lo estudiemos y 
                    observemos.}
            \item {Este programa funciona siempre que el proyecto no esté 
                    guardado en Scratch dentro de un "studio", por lo que debemos
                    mirar si podríamos modificarlo para evitar dicho error.}
            \item {Para que tarde menos en descargarse el proyecto podríamos
                    modificarlo para que únicamente se descargue el fichero
                    .sb2 evitando la descarga de imágenes o sonidos.}
    \end{itemize}

    \item{\textbf {Funcionalidad para Organizaciones:}}
    \begin{itemize}
        \item {El botón ha sido añadido en la parte superior de la página de inicio
                en color rojo (distinto al resto) para diferenciarlo.}
        \item {Al clicar en el enlace de "Organizaciones" conduce a una nueva 
                página para que el usuario sea consciente de ello, en el que 
                se explican las funcionalidades que pretendemos ofrecer.}
        \item {Estamos trabajando en dicha información y su traducción.}
    \end{itemize}


\end{enumerate}

\vspace{2cm}
\textbf{{\LARGE Revisión de números alcanzados}}
\vspace{0.5cm}

\begin{enumerate}  
    \item {\textbf{Plataforma de Dr.Scratch:}}
        \begin{itemize}
            \item {Número de proyectos analizados: 6610 proyectos.}
            \item {Número de lenguas de la plataforma: 2 (inglés y castellano).}
            \item {Número de países que visitan la plataforma: 10 países.}
            \item {Número de sesiones: 2669.}
            \item {Número de usuarios: 2096.}
            \item {Número de páginas vistas: 7477.}
        \end{itemize}
    \item {\textbf{Twitter:}}
        \begin{itemize}
            \item {Número de tweets: 11.}
            \item {Visitas al perfil:456.}
            \item {Menciones:18.}
            \item {Nuevos seguidores:58.}
        \end{itemize}
    \item {\textbf{Blog:}}
        \begin{itemize}
            \item {Visitas: 289.}
            \item {Visitantes: 170.}
            \item {Entradas publicadas: 4.}
        \end{itemize}

\end{enumerate}

\end{letter}

\end{document}
