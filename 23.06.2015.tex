\documentclass[a4paper,12pt]{letter}
\usepackage[spanish]{babel}
\usepackage[utf8]{inputenc}
\usepackage{geometry}
\usepackage{url}
\usepackage{calc}
\usepackage{setspace}
\usepackage{fixltx2e}
\usepackage{graphicx}
\usepackage{multicol}
\usepackage[normalem]{ulem}
%% Please revise the following command, if your babel
%% package does not support es-ES
\usepackage{color}
\usepackage{hyperref}

\pagestyle{empty}
%\makelabels
\topmargin -1.0 cm
\textheight 20.5 cm
\geometry{top=2cm, left=3cm, right=3cm, bottom=3cm}
 
\begin{document}


\begin{letter}
{}

\fbox{
    \parbox{15cm}{
        \setlength{\baselineskip}{18pt}
        \textbf{\textcolor[rgb]{0.000,0.502,0.502}{{\LARGE Acta de reunión de 
                Dr.Scratch:}}}\\
        \setlength{\baselineskip}{0pt}
        \textbf{\textcolor[rgb]{0.000,0.502,0.502}{{\LARGE Fomentando la 
                    creatividad y vocaciones científicas con Scratch}}}\\    
    }
}\\

\vspace{1cm}
\setlength{\parskip}{0.5mm}
\parbox{70mm}{
    \textbf{
        \textcolor[rgb]{0.000,0.502,0.502}{\large Asistentes:}
        \begin{itemize}
            \setlength{\parskip}{0.5mm}
            \item{\textcolor[rgb]{0.502,0.502,0.502}{Gregorio Robles}}
            \item{\textcolor[rgb]{0.502,0.502,0.502}{Jesús Moreno}}
            \item{\textcolor[rgb]{0.502,0.502,0.502}{Eva Hu}}
            \item{\textcolor[rgb]{0.502,0.502,0.502}{Mari Luz Aguado}}
        \end{itemize}
    }
}
\hfill
\parbox{90mm}{
    \bf{
        \textcolor[rgb]{0.000,0.502,0.502}{\large Entidad:}
        \textcolor[rgb]{0.502,0.502,0.502}{Universidad Rey Juan Carlos}\\ 
        \textcolor[rgb]{0.000,0.502,0.502}{\large Fecha:}
        \textcolor[rgb]{0.502,0.502,0.502}{23 de junio de 2015} \\
        \textcolor[rgb]{0.000,0.502,0.502}{\large Lugar:}\\
        \textcolor[rgb]{0.502,0.502,0.502}{Camino del Molino s/n \\
                                           28943 Fuenlabrada (Madrid)} \\
      
  
   }
}



\vspace{0.75cm}
\textbf{{\LARGE Temas tratados}}
\vspace{0.5cm}


En la reunión realizada el \textbf {23 de junio de 2015} en el {\bf Campus de
Fuenlabrada de la Universidad Rey Juan Carlos} y dirigida por Dr.Gregorio Robles
se trataron los siguientes temas:\\

\begin{enumerate}
    
    \item{\textbf {Taller para docentes en Figueras:}}
    \begin{itemize}
            \item {El sábado 27 se realizará en Figueras un taller para docentes
                    en el que se presentará la herramienta a 16 docentes.}
            \item {Se les mostará las nuevas pantallas de información más 
                    simplificadas.}
            \item {Les realizaremos unas encuestas para observar sus impresiones.}
    \end{itemize}
    
    \item{\textbf {Nuevas funcionalidades implementadas:}}
    \begin{itemize}
            \item {Página en la que se da las gracias a los colaboradores.}
            \item {Se ha incluido una explicación breve de cómo analizar los 
                    proyectos, indicando que hay dos opciones (mediante la url o 
                    subiendo el proyecto.}
            \item {Además se han añadido unos modales en las palabras "url" y
                    "proyecto" para explicar qué es una url y cómo descargar
                    el proyecto Scratch al ordenador.}
    \end{itemize}

    \item{\textbf {Próximas funcionalidades de Dr.Scratch:}}
    \begin{itemize}
        \item {Sigue pendiente el tour con Bootstraptour para guiar la experiencia
                de los dashboards.}
        \item {Se está terminando la funcionalidad para organizaciones. Queda por
                implementar la exportación de los datos del análisis a un fichero
                .csv y guardarlo en la base de datos para que el usuario pueda volver
                a descargarlo si lo desea.}
        \item {Añadir en el dashboard una opción directa para dar opiniones de 
                la información mostrada.}
    \end{itemize}


\end{enumerate}

\end{letter}

\end{document}
