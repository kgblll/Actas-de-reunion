\documentclass[a4paper,12pt]{letter}
\usepackage[spanish]{babel}
\usepackage[utf8]{inputenc}
\usepackage{geometry}
\usepackage{url}
\usepackage{calc}
\usepackage{setspace}
\usepackage{fixltx2e}
\usepackage{graphicx}
\usepackage{multicol}
\usepackage[normalem]{ulem}
%% Please revise the following command, if your babel
%% package does not support es-ES
\usepackage{color}
\usepackage{hyperref}

\pagestyle{empty}
%\makelabels
\topmargin -1.0 cm
\textheight 20.5 cm
\geometry{top=2cm, left=3cm, right=3cm, bottom=3cm}
 
\begin{document}


\begin{letter}
{}

\fbox{
    \parbox{15cm}{
        \setlength{\baselineskip}{18pt}
        \textbf{\textcolor[rgb]{0.000,0.502,0.502}{{\LARGE Acta de reunión de 
                Dr.Scratch:}}}\\
        \setlength{\baselineskip}{0pt}
        \textbf{\textcolor[rgb]{0.000,0.502,0.502}{{\LARGE Fomentando la 
                    creatividad y vocaciones científicas con Scratch}}}\\    
    }
}\\

\vspace{1cm}
\setlength{\parskip}{0.5mm}
\parbox{70mm}{
    \textbf{
        \textcolor[rgb]{0.000,0.502,0.502}{\large Asistentes:}
        \begin{itemize}
            \setlength{\parskip}{0.5mm}
            \item{\textcolor[rgb]{0.502,0.502,0.502}{Gregorio Robles}}
            \item{\textcolor[rgb]{0.502,0.502,0.502}{Jesús Moreno}}
            \item{\textcolor[rgb]{0.502,0.502,0.502}{Eva Hu}}
            \item{\textcolor[rgb]{0.502,0.502,0.502}{Mari Luz Aguado}}
        \end{itemize}
    }
}
\hfill
\parbox{90mm}{
    \bf{
        \textcolor[rgb]{0.000,0.502,0.502}{\large Entidad:}
        \textcolor[rgb]{0.502,0.502,0.502}{Universidad Rey Juan Carlos}\\ 
        \textcolor[rgb]{0.000,0.502,0.502}{\large Fecha:}
        \textcolor[rgb]{0.502,0.502,0.502}{30 de abril de 2015} \\
        \textcolor[rgb]{0.000,0.502,0.502}{\large Lugar:}\\
        \textcolor[rgb]{0.502,0.502,0.502}{Camino del Molino s/n \\
                                           28943 Fuenlabrada (Madrid)} \\
      
  
   }
}

\vspace{1cm}
\textbf{{\LARGE Temas tratados}}

\vspace{0.5cm}


En la reunión realizada el \textbf {30 de abril de 2015} en el {\bf Campus de
Fuenlabrada de la Universidad Rey Juan Carlos} y dirigida por Dr.Gregorio Robles
se trataron los siguientes temas:\\

\begin{enumerate}
    
    \item {\textbf {Lenguas co-oficiales en Dr.Scratch:}}
    \begin{itemize}
        \item {Actualmente el nº de lenguas disponibles en Dr.Scratch son dos: inglés y español.}
        \item {Se propone terminar de traducir las pantallas que explican cada una de las habilidades de programación y congelar el entorno de producción durante tres meses.}
        \item {Nos pondremos en contacto con profesores de diferentes regiones de España para que nos ayuden a traducir a lenguas co-oficiales.}
    \end{itemize}

    \item{\textbf {Comunidad de aprendizaje online:}}
    \begin{itemize}
        \item {Se tratará de la propia red social que tendrá Dr.Scratch en un futuro.}
        \item {En ésta comunidad se compartirán proyectos, puntuaciones, comentarios, etc.}

    \end{itemize}

	\item{\textbf {Materiales para docentes:}}
    \begin{itemize}
        \item {Se subirá al blog de Dr.Scratch un documento con los siguientes apartados:}
        \begin{itemize}
            \item {Introducción a Scratch.}
            \item {¿Cómo ir subiendo de puntuación?}
            \item {Ejercicios} 
            \item {Proyectos de ejemplo.} 
        \end{itemize}
        \item {Se deberá controlar el número de descargas.}
    \end{itemize}
        
  
    \vspace{0.5cm}
	\item {\textbf {Registro:}}
    \begin{itemize}
        \item {Se comenzará a desarrollar el registro de usuarios.}
        \item {Se solicitará:}
        \begin{itemize}
        \item {Nombre de usuario.}
        \item {Contraseña.}
        \item {Email. Si es menor de edad, solicitar el de sus padres.}
    \end{itemize}
    \item {Inicialmente los usuarios se podrán registrar mediante invitación o identificador.}
    \end{itemize}

    \item {\textbf Otros:}
        \begin{itemize}
            \item {Revisión de nuevas pantallas. Faltaría terminar las pantallas modales.}
            \item {Se ha desarrollado un script que mide el rendimiento del GetSb2 instalado en la máquina virtual de Azure.}
        \end{itemize}
\end{enumerate}



\end{letter}

\end{document}
