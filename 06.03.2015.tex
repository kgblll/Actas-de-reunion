\documentclass[a4paper,12pt]{letter}
\usepackage[spanish]{babel}
\usepackage[utf8]{inputenc}
\usepackage{geometry}
\usepackage{url}
\usepackage{calc}
\usepackage{setspace}
\usepackage{fixltx2e}
\usepackage{graphicx}
\usepackage{multicol}
\usepackage[normalem]{ulem}
%% Please revise the following command, if your babel
%% package does not support es-ES
\usepackage{color}
\usepackage{hyperref}

\pagestyle{empty}
%\makelabels
\topmargin -1.0 cm
\textheight 20.5 cm
\geometry{top=2cm, left=3cm, right=3cm, bottom=3cm}
 
\begin{document}


\begin{letter}
{}

\fbox{
    \parbox{15cm}{
        \setlength{\baselineskip}{18pt}
        \textbf{\textcolor[rgb]{0.000,0.502,0.502}{{\LARGE Acta de reunión de 
                Dr.Scratch:}}}\\
        \setlength{\baselineskip}{0pt}
        \textbf{\textcolor[rgb]{0.000,0.502,0.502}{{\LARGE Fomentando la 
                    creatividad y vocaciones científicas con Scratch}}}\\    
    }
}\\

\vspace{1cm}
\setlength{\parskip}{0.5mm}
\parbox{70mm}{
    \textbf{
        \textcolor[rgb]{0.000,0.502,0.502}{\large Asistentes:}
        \begin{itemize}
            \setlength{\parskip}{0.5mm}
            \item{\textcolor[rgb]{0.502,0.502,0.502}{Gregorio Robles}}
            \item{\textcolor[rgb]{0.502,0.502,0.502}{Jesús Moreno}}
            \item{\textcolor[rgb]{0.502,0.502,0.502}{Eva Hu}}
            \item{\textcolor[rgb]{0.502,0.502,0.502}{Mari Luz Aguado}}
        \end{itemize}
    }
}
\hfill
\parbox{90mm}{
    \bf{
        \textcolor[rgb]{0.000,0.502,0.502}{\large Entidad:}
        \textcolor[rgb]{0.502,0.502,0.502}{Universidad Rey Juan Carlos}\\ 
        \textcolor[rgb]{0.000,0.502,0.502}{\large Fecha:}
        \textcolor[rgb]{0.502,0.502,0.502}{6 de marzo de 2015} \\
        \textcolor[rgb]{0.000,0.502,0.502}{\large Lugar:}\\
        \textcolor[rgb]{0.502,0.502,0.502}{Camino del Molino s/n \\
                                           28943 Fuenlabrada (Madrid)} \\
      
  
   }
}



\vspace{0.75cm}
\textbf{{\LARGE Temas tratados}}
\vspace{0.5cm}


En la reunión realizada el \textbf {6 de marzo de 2015} en el {\bf Campus de
Fuenlabrada de la Universidad Rey Juan Carlos} y dirigida por Dr.Gregorio Robles
se trataron los siguientes temas:\\

\begin{enumerate}
    
    \item {\textbf {Comentar las últimas modificaciones realizadas en Dr.Scratch:}}
    \begin{itemize}
        \item {Solucionado el problema de la base de datos, ahora todos los proyectos se guardan correctamente en ella.}
    \end{itemize}

    \item{\textbf {Migración a Azure:}}
    \begin{itemize}
        \item {Necesario migrar el getsb2 a Azure para comprobar si el problema observado en el taller es debido al uso de la funcionalidad mediante Heroku y este no permite realizar tantas llamadas simultáneas.}
        \item {Mediante Visual Studio es posible publicar websites en Azure unidos a una Base de Datos, aprendiendo a realizarlo.}
    \end{itemize}
	\item{\textbf {Planteamiento de nuevas modificaciones en la plataforma de Dr.Scratch:}}
    \begin{itemize}
        \item {Poder imprimir diplomas en cada nivel.}
    \end{itemize}

\end{enumerate}

\vspace{0.75cm}
\textbf{{\LARGE Observaciones del taller para docentes realizado el 27 de febrero en Medialab Prado}}
\vspace{0.5cm}

\begin{enumerate}
    \item {Los docentes consideran necesario que mostremos información de cada una de las habilidades de programación, algo similar a las transparecias que mostró Jesús Moreno.}
    \item {Nos indicaron que los conceptos de paralelismo, flujo de control, abstracción... pueden ser difíciles de entender para sus estudiantes.}
    \item {Poder imprimir los resultados obtenidos.}
    \item {Nos preguntaron si este proyecto se podría trasladar a otros lenguajes.}
    \item {Les gustaría poder modificar la informaicón que verán sus alumnos, para según sus capacidades mostrarles más o menos información.}
\end{enumerate}

\vspace{0.75cm}
\textbf{{\LARGE Revisión de números alcanzados}}
\vspace{0.5cm}

\begin{enumerate}
    \item {Plataforma web (y aplicación móvil) de Dr.Scratch:}
        \begin{itemize}
            \item {Proyectos analizados hasta febrero: 2857 proyectos.}
            \item {Número de lenguas de la plataforma: 2 (inglés y castellano).}
        \end{itemize}
    \item {Twitter:}
        \begin{itemize}
            \item {Número de tweets: 3.}
            \item {Visitas al perfil: 567.}
            \item {Menciones: 48.}
            \item {Nuevos seguidores: 52.}
        \end{itemize}
    
\end{enumerate}


\end{letter}

\end{document}
