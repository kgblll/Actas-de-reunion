\documentclass[a4paper,12pt]{letter}
\usepackage[spanish]{babel}
\usepackage[utf8]{inputenc}
\usepackage{geometry}
\usepackage{url}
\usepackage{calc}
\usepackage{setspace}
\usepackage{fixltx2e}
\usepackage{graphicx}
\usepackage{multicol}
\usepackage[normalem]{ulem}
%% Please revise the following command, if your babel
%% package does not support es-ES
\usepackage{color}
\usepackage{hyperref}

\pagestyle{empty}
%\makelabels
\topmargin -1.0 cm
\textheight 20.5 cm
\geometry{top=2cm, left=3cm, right=3cm, bottom=3cm}
 
\begin{document}


\begin{letter}
{}

\fbox{
    \parbox{15cm}{
        \setlength{\baselineskip}{18pt}
        \textbf{\textcolor[rgb]{0.000,0.502,0.502}{{\LARGE Acta de reunión de 
                Dr.Scratch:}}}\\
        \setlength{\baselineskip}{0pt}
        \textbf{\textcolor[rgb]{0.000,0.502,0.502}{{\LARGE Fomentando la 
                    creatividad y vocaciones científicas con Scratch}}}\\    
    }
}\\

\vspace{1cm}
\setlength{\parskip}{0.5mm}
\parbox{70mm}{
    \textbf{
        \textcolor[rgb]{0.000,0.502,0.502}{\large Asistentes:}
        \begin{itemize}
            \setlength{\parskip}{0.5mm}
            \item{\textcolor[rgb]{0.502,0.502,0.502}{Gregorio Robles}}
            \item{\textcolor[rgb]{0.502,0.502,0.502}{Jesús Moreno}}
            \item{\textcolor[rgb]{0.502,0.502,0.502}{Eva Hu}}
            \item{\textcolor[rgb]{0.502,0.502,0.502}{Mari Luz Aguado}}
        \end{itemize}
    }
}
\hfill
\parbox{90mm}{
    \bf{
        \textcolor[rgb]{0.000,0.502,0.502}{\large Entidad:}
        \textcolor[rgb]{0.502,0.502,0.502}{Universidad Rey Juan Carlos}\\ 
        \textcolor[rgb]{0.000,0.502,0.502}{\large Fecha:}
        \textcolor[rgb]{0.502,0.502,0.502}{15 de abril de 2015} \\
        \textcolor[rgb]{0.000,0.502,0.502}{\large Lugar:}\\
        \textcolor[rgb]{0.502,0.502,0.502}{Camino del Molino s/n \\
                                           28943 Fuenlabrada (Madrid)} \\
      
  
   }
}



\vspace{0.75cm}
\textbf{{\LARGE Temas tratados}}
\vspace{0.5cm}


En la reunión realizada el \textbf {15 de abril de 2015} en el {\bf Campus de
Fuenlabrada de la Universidad Rey Juan Carlos} y dirigida por Dr.Gregorio Robles
se trataron los siguientes temas:\\

\begin{enumerate}
    
    \item{\textbf {Se comentaron las observaciones realizadas en el taller para alumnos realizado el día 9 de abril, comentadas más adelante.}}
    
    \item{\textbf {Registro en Google Analytics:}}
    \begin{itemize}
            \item {Realizado el registro en Google Analytics y añadido el código javascript suministrado para poder empezar a realizar el seguimiento de Dr.Scratch.}
    \end{itemize}

    \item{\textbf {Migración a Azure:}}
    \begin{itemize}
        \item {Solucionado el error del getsb2, funcionando correctamente y pendiente de realización de la prueba de rendimiento.}
        \item {Pendiente de respuesta desde Microsoft al problema de hairball.}
    \end{itemize}

	\item{\textbf {Planteamiento de nuevas modificaciones en la plataforma de Dr.Scratch:}}
    \begin{itemize}
        \item {Revisión de la traducción en las pantallas de aprendizaje.}
        \item {Modificación de las pantallas de información para incorporar las necesidades observadas.}
    \end{itemize}

    \item{\textbf {Blog de Dr.Scratch:}}
    \begin{itemize}
        \item {Entrada del taller de alumnos en el CEIP Lope de Vega realizado el 9 de abril de 2015.}
        \item {Pendiente de realizar una entrada para el Scratch Day.}
        \item {Realizaremos una entrada por cada habilidad explicándola brevemente.}
    \end{itemize}

\end{enumerate}

\vspace{2cm}
\textbf{{\LARGE Observaciones realizadas en el taller para alumnos}}
\vspace{0.5cm}


    En el taller realizado el día \textbf {9 de abril de 2015} en el \textbf {CEIP Lope de Vega} en Madrid con niños de 5º y 6º de primaria, entre 10 y 12 años, pudimos obtener las siguientes conclusiones observando cómo los alumnos utilizaban la herramienta:
\begin{enumerate}  
    \item {Únicamente observan la primera parte de la pantalla principal (no utilizan el scroll del ratón). Esto no supone demasiado problema, ya que la información que se muestra en la parte inferior está más destina a docentes.}
    \item {Un problema observado fue que este grupo de alumnos en general no sabían que es una url por lo que se sentían confusos cuando esta información se les pedía.}
    \item {No saben descargarse los proyectos de Scratch a sus ordenadores para subirlos a la página web de Dr.Scratch.}
    \item {Debemos separar los dos tipos de análisis porque no les queda claro cómo utilizarlo.}
    \item {En las páginas donde Dr.Scratch muestra la información acerca del análisis del proyecto hay unos carrusels con información que no les llama la atención.}
    \item {Les gusta mucho ver qué nivel tienen y la nota pero prefieren que el 21 sea más grande y redondeado, como 30.}
    \item {Sería mejor que sólo les salga "aprende más" para mostrarles información adicional en aquellas habilidades que tengan menor puntuación, porque sino no saben qué hacer con tanta información.}
\end{enumerate}

\end{letter}

\end{document}
