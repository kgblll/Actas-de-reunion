\documentclass[a4paper,12pt]{letter}
\usepackage[spanish]{babel}
\usepackage[utf8]{inputenc}
\usepackage{geometry}
\usepackage{url}
\usepackage{calc}
\usepackage{setspace}
\usepackage{fixltx2e}
\usepackage{graphicx}
\usepackage{multicol}
\usepackage[normalem]{ulem}
%% Please revise the following command, if your babel
%% package does not support es-ES
\usepackage{color}
\usepackage{hyperref}

\pagestyle{empty}
%\makelabels
\topmargin -1.0 cm
\textheight 20.5 cm
\geometry{top=2cm, left=3cm, right=3cm, bottom=3cm}
 
\begin{document}


\begin{letter}
{}

\fbox{
    \parbox{15cm}{
        \setlength{\baselineskip}{18pt}
        \textbf{\textcolor[rgb]{0.000,0.502,0.502}{{\LARGE Acta de reunión de 
                Dr.Scratch:}}}\\
        \setlength{\baselineskip}{0pt}
        \textbf{\textcolor[rgb]{0.000,0.502,0.502}{{\LARGE Fomentando la 
                    creatividad y vocaciones científicas con Scratch}}}\\    
    }
}\\

\vspace{1cm}
\setlength{\parskip}{0.5mm}
\parbox{70mm}{
    \textbf{
        \textcolor[rgb]{0.000,0.502,0.502}{\large Asistentes:}
        \begin{itemize}
            \setlength{\parskip}{0.5mm}
            \item{\textcolor[rgb]{0.502,0.502,0.502}{Gregorio Robles}}
            \item{\textcolor[rgb]{0.502,0.502,0.502}{Jesús Moreno}}
            \item{\textcolor[rgb]{0.502,0.502,0.502}{Eva Hu}}
            \item{\textcolor[rgb]{0.502,0.502,0.502}{Mari Luz Aguado}}
        \end{itemize}
    }
}
\hfill
\parbox{90mm}{
    \bf{
        \textcolor[rgb]{0.000,0.502,0.502}{\large Entidad:}
        \textcolor[rgb]{0.502,0.502,0.502}{Universidad Rey Juan Carlos}\\ 
        \textcolor[rgb]{0.000,0.502,0.502}{\large Fecha:}
        \textcolor[rgb]{0.502,0.502,0.502}{19 de junio de 2015} \\
        \textcolor[rgb]{0.000,0.502,0.502}{\large Lugar:}\\
        \textcolor[rgb]{0.502,0.502,0.502}{Camino del Molino s/n \\
                                           28943 Fuenlabrada (Madrid)} \\
      
  
   }
}



\vspace{0.75cm}
\textbf{{\LARGE Temas tratados}}
\vspace{0.5cm}


En la reunión realizada el \textbf {19 de junio de 2015} en el {\bf Campus de
Fuenlabrada de la Universidad Rey Juan Carlos} y dirigida por Dr.Gregorio Robles
se trataron los siguientes temas:\\

\begin{enumerate}
    
    \item{\textbf {Talleres para alumnos:}}
    \begin{itemize}
            \item {En los talleres realizados el día 18 de junio en el CEIP
                    Gonzalo Fernandez de Córdoba se realizaron dos talleres con
                    alumnos de 4º de primaria. En una de las clases conocían 
                    Scratch y les dejamos utilizar Dr.Scratch de forma independiente,
                    en el otro taller no conocían Scratch por lo que los guiamos
                    mucho más.}
            \item {Obtuvimos como conclusión que la apariencia de las nuevas
                    pantallas les gustaba más, pero no saben qué hacer
                    con la información, por lo que necesitan algún tipo de guía.}
            \item {Señalar que las barras de progreso de las distintas habilidades
                    les llama mucho la atención y debemos añadir algo así en la
                    puntuación total, ahora se fijan menos en ella.}
    \end{itemize}
    
    \item{\textbf {Dominio talleres.drscratch.org:}}
    \begin{itemize}
            \item {Creamos un dominio nuevo llamado "talleres.drscratch.org" que
                    también está vinculado con la IP de Azure, al igual que 
                    "www.drscratch.org" para utilizarla en los talleres.}
            \item {Tenemos problemas configurando Apache de forma que cada uno
                    tenga como path una carpeta distinta.}
    \end{itemize}

    \item{\textbf {Próximas funcionalidades de Dr.Scratch:}}
    \begin{itemize}
        \item {Una página en la que se de las gracias a los colaboradores de 
                las traducciones y de los colegios donde estamos realizando talleres
                para agradecer sus contribuciones.}
        \item {Guiar la experiencia de los dashboards mediante Bootstraptour, 
                para solucionar el problema de su lectura.}
    \end{itemize}


\end{enumerate}

\end{letter}

\end{document}
