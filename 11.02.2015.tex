\documentclass[a4paper,12pt]{letter}
\usepackage[spanish]{babel}
\usepackage[utf8]{inputenc}
\usepackage{geometry}
\usepackage{url}
\usepackage{calc}
\usepackage{setspace}
\usepackage{fixltx2e}
\usepackage{graphicx}
\usepackage{multicol}
\usepackage[normalem]{ulem}
%% Please revise the following command, if your babel
%% package does not support es-ES
\usepackage{color}
\usepackage{hyperref}

\pagestyle{empty}
%\makelabels
\topmargin -1.0 cm
\textheight 20.5 cm
\geometry{top=2cm, left=3cm, right=3cm, bottom=3cm}
 
\begin{document}


\begin{letter}
{}

\fbox{
    \parbox{15cm}{
        \setlength{\baselineskip}{18pt}
        \textbf{\textcolor[rgb]{0.000,0.502,0.502}{{\LARGE Acta de reunión de 
                Dr.Scratch:}}}\\
        \setlength{\baselineskip}{0pt}
        \textbf{\textcolor[rgb]{0.000,0.502,0.502}{{\LARGE Fomentando la 
                    creatividad y vocaciones científicas con Scratch}}}\\    
    }
}\\

\vspace{1cm}
\setlength{\parskip}{0.5mm}
\parbox{70mm}{
    \textbf{
        \textcolor[rgb]{0.000,0.502,0.502}{\large Asistentes:}
        \begin{itemize}
            \setlength{\parskip}{0.5mm}
            \item{\textcolor[rgb]{0.502,0.502,0.502}{Gregorio Robles}}
            \item{\textcolor[rgb]{0.502,0.502,0.502}{Jesús Moreno}}
            \item{\textcolor[rgb]{0.502,0.502,0.502}{Eva Hu}}
            \item{\textcolor[rgb]{0.502,0.502,0.502}{Mari Luz Aguado}}
        \end{itemize}
    }
}
\hfill
\parbox{90mm}{
    \bf{
        \textcolor[rgb]{0.000,0.502,0.502}{\large Entidad:}
        \textcolor[rgb]{0.502,0.502,0.502}{Universidad Rey Juan Carlos}\\ 
        \textcolor[rgb]{0.000,0.502,0.502}{\large Fecha:}
        \textcolor[rgb]{0.502,0.502,0.502}{11 de febrero de 2015} \\
        \textcolor[rgb]{0.000,0.502,0.502}{\large Lugar:}\\
        \textcolor[rgb]{0.502,0.502,0.502}{Camino del Molino s/n \\
                                           28943 Fuenlabrada (Madrid)} \\
      
  
   }
}

\vspace{0.75cm}
\textbf{{\LARGE Temas tratados}}

\vspace{0.5cm}


En la reunión realizada el \textbf {11 de enero de 2015} en el {\bf Campus de
Fuenlabrada de la Universidad Rey Juan Carlos} y dirigida por Dr.Gregorio Robles
se trataron los siguientes temas:\\

\begin{enumerate}
    
    \item {\textbf {Comentar las últimas modificaciones realizadas en Dr.Scratch:}}
    \begin{itemize}
        \item {Instalación y puesta en marcha de la base de datos de Msql en
                lugar de sqlite en la máquina de preproducción.}
        \item {Detectamos que los análisis realizados mediante url no se estaban
                registrando bien en la base de datos ni en el servidor. 
                Hemos normalizado el código de las subidas por proyecto y por 
                url, para que el código sea similar y solucionar el problema.}
        \item {Traducción al Castellano de la página de inicio de Dr.Scratch, 
                pendiente de revisión por Jesús Moreno.}
    \end{itemize}

    \item{\textbf {Migración a Azure:}}
    \begin{itemize}
        \item {Creación de cuentas de Microsoft para comenzar a ver Azure antes
                de acudir al curso impartido por Microsoft el 12.02.2015. Vamos
                a crear un website donde situar nuestra plataforma. Jesús
                explica al personal de apoyo la interfaz de Azure.}

    \end{itemize}

	\item{\textbf {Planteamiento de nuevas modificaciones en la plataforma de 
                    Dr.Scratch:}}
    \begin{itemize}
        \item {Crear nuevos dashboards que se parezcan más a la nueva apariencia,
                y evitar la sensación de cambiar de sitio web.}
        \item {Crear un dashboards de estadísticas de los proyectos analizados,
               incluyendo aspectos como: el número de proyectos analizados,
                el avance de nivel de los proyectos que se analizan, estadísticas
                de las habilidades de pensamiento computacional...}
    \end{itemize}

    \item {\textbf {Base de datos:}}
        \begin{itemize}
            \item {Hacer pruebas en preproducción para asegurar que funciona
                    correctamente Mysql para pasarlo a producción.}
        \end{itemize}
\end{enumerate}

\end{letter}

\end{document}
