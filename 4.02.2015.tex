\documentclass[a4paper,12pt]{letter}
\usepackage[spanish]{babel}
\usepackage[utf8]{inputenc}
\usepackage{geometry}
\usepackage{url}
\usepackage{calc}
\usepackage{setspace}
\usepackage{fixltx2e}
\usepackage{graphicx}
\usepackage{multicol}
\usepackage[normalem]{ulem}
%% Please revise the following command, if your babel
%% package does not support es-ES
\usepackage{color}
\usepackage{hyperref}

\pagestyle{empty}
%\makelabels
\topmargin -1.0 cm
\textheight 20.5 cm
\geometry{top=2cm, left=3cm, right=3cm, bottom=3cm}
 
\begin{document}


\begin{letter}
{}

\fbox{
    \parbox{15cm}{
        \setlength{\baselineskip}{18pt}
        \textbf{\textcolor[rgb]{0.000,0.502,0.502}{{\LARGE Acta de reunión de 
                Dr.Scratch:}}}\\
        \setlength{\baselineskip}{0pt}
        \textbf{\textcolor[rgb]{0.000,0.502,0.502}{{\LARGE Fomentando la 
                    creatividad y vocaciones científicas con Scratch}}}\\    
    }
}\\

\vspace{1cm}
\setlength{\parskip}{0.5mm}
\parbox{70mm}{
    \textbf{
        \textcolor[rgb]{0.000,0.502,0.502}{\large Asistentes:}
        \begin{itemize}
            \setlength{\parskip}{0.5mm}
            \item{\textcolor[rgb]{0.502,0.502,0.502}{Gregorio Robles}}
            \item{\textcolor[rgb]{0.502,0.502,0.502}{Jesús Moreno}}
            \item{\textcolor[rgb]{0.502,0.502,0.502}{Eva Hu}}
            \item{\textcolor[rgb]{0.502,0.502,0.502}{Mari Luz Aguado}}
        \end{itemize}
    }
}
\hfill
\parbox{90mm}{
    \bf{
        \textcolor[rgb]{0.000,0.502,0.502}{\large Entidad:}
        \textcolor[rgb]{0.502,0.502,0.502}{Universidad Rey Juan Carlos}\\ 
        \textcolor[rgb]{0.000,0.502,0.502}{\large Fecha:}
        \textcolor[rgb]{0.502,0.502,0.502}{4 de febrero de 2015} \\
        \textcolor[rgb]{0.000,0.502,0.502}{\large Lugar:}\\
        \textcolor[rgb]{0.502,0.502,0.502}{Camino del Molino s/n \\
                                           28943 Fuenlabrada (Madrid)} \\
      
  
   }
}

\vspace{0.75cm}
\textbf{{\LARGE Temas tratados}}

\vspace{0.5cm}


En la reunión realizada el \textbf {4 de febrero de 2015} en el {\bf Campus de
Fuenlabrada de la Universidad Rey Juan Carlos} y dirigida por Dr.Gregorio Robles
se trataron los siguientes temas:\\

\begin{enumerate}
    
    \item {\textbf {Comentar las últimas modificaciones realizadas en Dr.Scratch:}}
    \begin{itemize}
        \item {Introducción del texto en inglés a la nueva apariencia diseñada por José Ignacio Huertas, mucho más profesional y permite explicar qué es Dr.Scratch, cómo utilizarlo y las nuevas funcionalidades que queremos incluir, además de mostrar diversas formas para contactar con el equipo..}
        \item {Análisis mediante url.}
        \item {Implementación que permite mostrar errores si pulsan el botón de analizar sin haber subido ningún fichero .sb2 o sin introducir url.}
    \end{itemize}

    \item{\textbf {Migración a Azure:}}
    \begin{itemize}
        \item {El personal de apoyo se inscribe en un curso impartido por Microsoft para tratar de migrar la plataforma a Azure y evitar problemas de caídas de los servidores de la Universidad..}

    \end{itemize}

	\item{\textbf {Planteamiento de nuevas modificaciones en la plataforma de 
                    Dr.Scratch:}}
    \begin{itemize}
        \item {Un plug-in que permita acceder y analizar proyectos directamente desde Scratch.}
        \item {Traducción de la plataforma al Castellano.}
    \end{itemize}

    \item {\textbf {Apariencia de Dr.Scratch:}}
        \begin{itemize}
            \item {Según las condiciones de uso de Dr.Scratch está permitido usar su logo pero de momento no lo vamos a hacer.}
            \item {Pensar para la siguiente reunión modificaciones en el dashboard.}
        \end{itemize}

    \item {\textbf {Base de datos:}}
        \begin{itemize}
            \item {Actualmente estamos trabajando con sqlite, hay que cambiar a mysql.}
        \end{itemize}
\end{enumerate}

\textbf{{\LARGE Revisión de números alcanzados}}


\begin{enumerate}

    \item {\textbf {Plataforma web (y aplicación móvil) de Dr.Scratch:}}
        \begin{itemize}
            \item {Protectos analizados hasta enero: 2121 proyectos.}
            \item {Número de lenguas de la plataforma: 1 lengua (inglés).}
        \end{itemize}


    \item {\textbf {Twitter:}}
        \begin{itemize}
            \item {Número de tweets:14.}
            \item {Visitas al perfil:477.}
            \item {Menciones:15.}
            \item {Nuevos seguidores:83.}
        \end{itemize}


\end{enumerate}

\end{letter}

\end{document}
