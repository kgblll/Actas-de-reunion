\documentclass[a4paper,12pt]{letter}
\usepackage[spanish]{babel}
\usepackage[utf8]{inputenc}
\usepackage{geometry}
\usepackage{url}
\usepackage{calc}
\usepackage{setspace}
\usepackage{fixltx2e}
\usepackage{graphicx}
\usepackage{multicol}
\usepackage[normalem]{ulem}
%% Please revise the following command, if your babel
%% package does not support es-ES
\usepackage{color}
\usepackage{hyperref}

\pagestyle{empty}
%\makelabels
\topmargin -1.0 cm
\textheight 20.5 cm
\geometry{top=2cm, left=3cm, right=3cm, bottom=3cm}
 
\begin{document}


\begin{letter}
{}

\fbox{
    \parbox{15cm}{
        \setlength{\baselineskip}{18pt}
        \textbf{\textcolor[rgb]{0.000,0.502,0.502}{{\LARGE Acta de reunión de Dr.Scratch:}}}\\
        \setlength{\baselineskip}{0pt}
        \textbf{\textcolor[rgb]{0.000,0.502,0.502}{{\LARGE Fomentando la creatividad y vocaciones científicas con Scratch}}}\\    
    }
}\\

\vspace{1cm}
\setlength{\parskip}{0.5mm}
\parbox{70mm}{
    \textbf{
        \textcolor[rgb]{0.000,0.502,0.502}{\large Asistentes:}
        \begin{itemize}
            \setlength{\parskip}{0.5mm}
            \item{\textcolor[rgb]{0.502,0.502,0.502}{Gregorio Robles}}
            \item{\textcolor[rgb]{0.502,0.502,0.502}{Jesús Moreno}}
            \item{\textcolor[rgb]{0.502,0.502,0.502}{Eva Hu}}
            \item{\textcolor[rgb]{0.502,0.502,0.502}{Mari Luz Aguado}}
        \end{itemize}
    }
}
\hfill
\parbox{90mm}{
    \bf{
        \textcolor[rgb]{0.000,0.502,0.502}{\large Entidad:}
        \textcolor[rgb]{0.502,0.502,0.502}{Universidad Rey Juan Carlos}\\ 
        \textcolor[rgb]{0.000,0.502,0.502}{\large Fecha:}
        \textcolor[rgb]{0.502,0.502,0.502}{30 de enero de 2015} \\
        \textcolor[rgb]{0.000,0.502,0.502}{\large Lugar:}\\
        \textcolor[rgb]{0.502,0.502,0.502}{Parque Científico de la UC3M \\ 
                                           Avenida Gregorio Peces Barba, 1 \\
                                           (Polígono Industrial Legatec)\\
                                           28919 Leganés (Madrid)}\\
      
  
   }
}

\vspace{1cm}
\textbf{{\LARGE Temas tratados}}

\vspace{0.5cm}


En la reunión realizada el \textbf {30 de enero de 2015} en {\bf Parque 
Científico de la UC3M de Leganés} y dirigida por Dr.Gregorio Robles se trataron 
los siguientes temas:\\

\begin{enumerate}
    
    \item {\textbf {Comentar las últimas modificaciones realizadas en Dr.Scratch:}}
    \begin{itemize}
        \item {Cambio de la apariencia de la página principal.}
        \item {Introducción de una barra de progreso mientras se realiza el análisis del proyecto.}
    \end{itemize}

	\item{\textbf {Planteamiento de nuevas modificaciones en la plataforma de Dr.Scratch:}}
    \begin{itemize}
        \item {Introducción de una interfaz de apoyo para un profesor que quiera 
              seguir a sus alumnos.}
        \item {Traducción al castellano.}
    \end{itemize}

	\item{\textbf {Fijar la fecha de un taller para docentes el día 27 de febrero, 
        tareas necesarias:}}
    \begin{itemize}
        \item {Buscar lugar para realizarlo, posiblemente Medialab Prado. 
              Eva M. Castro se encarga de contactar con ellos.}
        \item {Contactar con docentes que puedan estar interesados. Tarea a 
              realizar por Jesús Moreno.}
        \item {Realización de certificados de asistencia, Eva Hu se presta 
              voluntaria para realizarlo.}
        \item {Nesaria la traducción de la plataforma a Castellano para dicho
               día, tarea asignada a Mari Luz Aguado.}
    \end{itemize}

    
	\item{\textbf {Migración a Azure:}}
    \begin{itemize}
        \item {Se plantea una fecha límite de una semana.}
    \end{itemize}
    \vspace{0.5cm}
	\item {\textbf {Material de apoyo a docentes:}}
    \begin{itemize}
        \item {José Ignacio Huertas y Jesús Moreno tienen material disponible 
              para dicha finalidad que se plantea compartir en la plataforma 
              Wordpress y a través de Twitter.}
        \item {Necesario pensar en una plataforma donde podamos contabilizar
               las descargas realizadas.}
    \end{itemize}

    \item {\textbf {Necesaria la creación de un blog en Wordpress de Dr.Scratch.}}

\end{enumerate}

\vspace{1cm} 
\textbf{\LARGE Revisión de números alcanzados}

\setlength{\parskip}{4mm}
\begin{enumerate}
    \item {\textbf{Plataforma web (y aplicación móvil) de Dr.Scratch:}}
	
    \setlength{\parskip}{2mm}
    \begin{itemize}
      	\setlength{\parskip}{0mm}        
       	\item {Aún no hay registro de usuarios.}      
	    \item {Proyectos analizados: 2121 proyectos.}
	    \item {Número de lenguas de la plataforma: 1 lengua (inglés).}
    \end{itemize}

    \item {\textbf{Twitter y blogs:}}

    \setlength{\parskip}{2mm}
    \begin{itemize}
      	\setlength{\parskip}{0mm}        
       	\item {Número de tweets: 30.}  
	    \item {Número de followers: 361.}
	    \item {Número de menciones en Twitter: 52.}
    \end{itemize}

\end{enumerate}
\end{letter}

\end{document}
