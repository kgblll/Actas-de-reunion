\documentclass[a4paper,12pt]{letter}
\usepackage[spanish]{babel}
\usepackage[utf8]{inputenc}
\usepackage{geometry}
\usepackage{url}
\usepackage{calc}
\usepackage{setspace}
\usepackage{fixltx2e}
\usepackage{graphicx}
\usepackage{multicol}
\usepackage[normalem]{ulem}
%% Please revise the following command, if your babel
%% package does not support es-ES
\usepackage{color}
\usepackage{hyperref}

\pagestyle{empty}
%\makelabels
\topmargin -1.0 cm
\textheight 20.5 cm
\geometry{top=2cm, left=3cm, right=3cm, bottom=3cm}
 
\begin{document}


\begin{letter}
{}

\fbox{
    \parbox{15cm}{
        \setlength{\baselineskip}{18pt}
        \textbf{\textcolor[rgb]{0.000,0.502,0.502}{{\LARGE Acta de reunión de 
                Dr.Scratch:}}}\\
        \setlength{\baselineskip}{0pt}
        \textbf{\textcolor[rgb]{0.000,0.502,0.502}{{\LARGE Fomentando la 
                    creatividad y vocaciones científicas con Scratch}}}\\    
    }
}\\

\vspace{1cm}
\setlength{\parskip}{0.5mm}
\parbox{70mm}{
    \textbf{
        \textcolor[rgb]{0.000,0.502,0.502}{\large Asistentes:}
        \begin{itemize}
            \setlength{\parskip}{0.5mm}
            \item{\textcolor[rgb]{0.502,0.502,0.502}{Gregorio Robles}}
            \item{\textcolor[rgb]{0.502,0.502,0.502}{Jesús Moreno}}
            \item{\textcolor[rgb]{0.502,0.502,0.502}{Eva Hu}}
            \item{\textcolor[rgb]{0.502,0.502,0.502}{Mari Luz Aguado}}
        \end{itemize}
    }
}
\hfill
\parbox{90mm}{
    \bf{
        \textcolor[rgb]{0.000,0.502,0.502}{\large Entidad:}
        \textcolor[rgb]{0.502,0.502,0.502}{Universidad Rey Juan Carlos}\\ 
        \textcolor[rgb]{0.000,0.502,0.502}{\large Fecha:}
        \textcolor[rgb]{0.502,0.502,0.502}{8 de abril de 2015} \\
        \textcolor[rgb]{0.000,0.502,0.502}{\large Lugar:}\\
        \textcolor[rgb]{0.502,0.502,0.502}{Camino del Molino s/n \\
                                           28943 Fuenlabrada (Madrid)} \\
      
  
   }
}



\vspace{0.75cm}
\textbf{{\LARGE Temas tratados}}
\vspace{0.5cm}


En la reunión realizada el \textbf {8 de abril de 2015} en el {\bf Campus de
Fuenlabrada de la Universidad Rey Juan Carlos} y dirigida por Dr.Gregorio Robles
se trataron los siguientes temas:\\

\begin{enumerate}
    
    \item {\textbf {Comentar las últimas modificaciones realizadas en Dr.Scratch:}}
    \begin{itemize}
        \item {Realizada correctamente la traducción del botón de Twitter, ya es posible compartir perfectamente la puntuación obtenida en Dr.Scratch en la cuenta de Twitter del usuario.}
        \item {Elaboradas nuevas páginas donde el usuario podrá aprender más acerca de los conceptos de paralelismo, abstracción, representación de la información...}
        \item {Implementado el análisis de proyectos de la versión 1.4 de Scratch.}
    \end{itemize}

    \item {\textbf {Taller de Dr.Scratch para alumnos:}}
    \begin{itemize}
        \item {Elaboración de cuestionario para entregar a los alumnos pidiéndoles información sobre qué aspectos de la página les gustan y cuáles modificarían.}
        \item {Pegatinas de Dr.Scratch para aquellos alumnos que completen el cuestionario.}
        \item {Certificado de asistencia al taller para alumnos de Dr.Scratch que se les entregará tras conseguir aumentar de nivel su proyecto en Scratch.}
    \end{itemize}

    \item{\textbf {Registro en Google Analytics para poder realizar un estudio de los usuarios que utilizan Dr.Scratch y obtener informaición del sitio web.}}

    \item{\textbf {Migración a Azure:}}
    \begin{itemize}
        \item {Funcionando getsb2 en la máquina virtual de Azure pero produce un error.}
        \item {Avance en la migración de hairball pero aún no muestra la salida correctamente.}
    \end{itemize}
	\item{\textbf {Planteamiento de nuevas modificaciones en la plataforma de Dr.Scratch:}}
    \begin{itemize}
        \item {Observaremos el comportamiento de los alumnos para decidir que modificaciones realizar.}
    \end{itemize}

    \item{\textbf {Blog de Dr.Scratch:}}
    \begin{itemize}
        \item {Apariencia del blog terminada.}
        \item {Entrada sobre el taller realizado el 27 de febrero en Medialab Prado.}
        \item {Entrada mostrando el video demostrativo de uso de Dr.Scratch presentándolo.}
        \item {Entrada comentando los errores típicos que todo programador realiza al principio.}
    \end{itemize}

\end{enumerate}

\vspace{0.75cm}
\textbf{{\LARGE Revisión de números alcanzados}}
\vspace{0.5cm}

\begin{enumerate}
    \item{\textbf {Plataforma web (y aplicación móvil) de Dr.Scratch}}
    \begin{itemize}
        \item {Proyectos analizados hasta marzo: 3158 proyectos.}
        \item {Número de lenguas de la plataforma: 2 (inglés y castellano).}
    \end{itemize}

    \item{\textbf {Twitter:}}
    \begin{itemize}
        \item {Número de tweets:7.}
        \item {Visitas al perfil:409.}
        \item {Menciones:20.}
        \item {Nuevos seguidores:38.}

    \end{itemize}

\end{enumerate}

\end{letter}

\end{document}
