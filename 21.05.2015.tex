\documentclass[a4paper,12pt]{letter}
\usepackage[spanish]{babel}
\usepackage[utf8]{inputenc}
\usepackage{geometry}
\usepackage{url}
\usepackage{calc}
\usepackage{setspace}
\usepackage{fixltx2e}
\usepackage{graphicx}
\usepackage{multicol}
\usepackage[normalem]{ulem}
%% Please revise the following command, if your babel
%% package does not support es-ES
\usepackage{color}
\usepackage{hyperref}

\pagestyle{empty}
%\makelabels
\topmargin -1.0 cm
\textheight 20.5 cm
\geometry{top=2cm, left=3cm, right=3cm, bottom=3cm}
 
\begin{document}


\begin{letter}
{}

\fbox{
    \parbox{15cm}{
        \setlength{\baselineskip}{18pt}
        \textbf{\textcolor[rgb]{0.000,0.502,0.502}{{\LARGE Acta de reunión de 
                Dr.Scratch:}}}\\
        \setlength{\baselineskip}{0pt}
        \textbf{\textcolor[rgb]{0.000,0.502,0.502}{{\LARGE Fomentando la 
                    creatividad y vocaciones científicas con Scratch}}}\\    
    }
}\\

\vspace{1cm}
\setlength{\parskip}{0.5mm}
\parbox{70mm}{
    \textbf{
        \textcolor[rgb]{0.000,0.502,0.502}{\large Asistentes:}
        \begin{itemize}
            \setlength{\parskip}{0.5mm}
            \item{\textcolor[rgb]{0.502,0.502,0.502}{Gregorio Robles}}
            \item{\textcolor[rgb]{0.502,0.502,0.502}{Jesús Moreno}}
            \item{\textcolor[rgb]{0.502,0.502,0.502}{Eva Hu}}
            \item{\textcolor[rgb]{0.502,0.502,0.502}{Mari Luz Aguado}}
        \end{itemize}
    }
}
\hfill
\parbox{90mm}{
    \bf{
        \textcolor[rgb]{0.000,0.502,0.502}{\large Entidad:}
        \textcolor[rgb]{0.502,0.502,0.502}{Universidad Rey Juan Carlos}\\ 
        \textcolor[rgb]{0.000,0.502,0.502}{\large Fecha:}
        \textcolor[rgb]{0.502,0.502,0.502}{21 de mayo de 2015} \\
        \textcolor[rgb]{0.000,0.502,0.502}{\large Lugar:}\\
        \textcolor[rgb]{0.502,0.502,0.502}{Camino del Molino s/n \\
                                           28943 Fuenlabrada (Madrid)} \\
      
  
   }
}

\vspace{1cm}
\textbf{{\LARGE Temas tratados}}

\vspace{0.5cm}


En la reunión realizada el \textbf {21 de mayo de 2015} en el {\bf Campus de
Fuenlabrada de la Universidad Rey Juan Carlos} y dirigida por Dr.Gregorio Robles
se trataron los siguientes temas:\\

\begin{enumerate}
    
    \item {\textbf {Google Rise Summit 2015:}}
    
    \begin{itemize}
    Se comentaron los aspectos más importantes recogidos en el evento Google Rise Summit 2015.
        \item {Dr.Scratch tuvo gran acogida entre los asistentes.}
        \item {Deberemos hacer comunidad con los usuarios, esto es, que participen activamente en el desarrollo de Dr.Scratch.}
    \end{itemize}

    \item{\textbf {Organizaciones:}}
    
    \begin{itemize}
    Como consecuencia del punto anterior, se ha priorizado el desarrollo de una nueva funcionalidad para Dr.Scratch: registro de organizaciones. Para ello deberemos hacer lo siguiente:
        \item {Crear apartado 'Organizaciones' para acceder a una nueva pantalla de registro y acceso de organizaciones.}
        \item {Al acceder a una organización, se podrá ver una pantalla de inicio de Dr.Scratch personalizada por los propios fundadores de la asociación.}
        \item {Deberá poder registrar a sus usuarios por identificador.}
         \item {Deberá poder recopilar los resultados obtenidos en los proyectos de cada usuario, y descargarlo en formato de hoja de cálculo(deberemos por lo tanto, guardar en la base de datos la puntuación obtenida).}

    \end{itemize}

	\item{\textbf {Feedback:}}
    \begin{itemize}
    Deberemos disponer de una funcionalidad que facilite a los usuarios enviarnos algún tipo de realimentación sobre la herramienta.
        \item {Utilizaremos un formulario.}
    \end{itemize}
\vspace{0.5cm}
    \item {\textbf Otros:}
        \begin{itemize}
            \item {Crear canal de Youtube.}
            \item {Recopilar enlaces de prensa donde Dr.Scratch ha sido mencionado.}
        \end{itemize}
\end{enumerate}



\end{letter}

\end{document}
