\documentclass[a4paper,12pt]{letter}
\usepackage[spanish]{babel}
\usepackage[utf8]{inputenc}
\usepackage{geometry}
\usepackage{url}
\usepackage{calc}
\usepackage{setspace}
\usepackage{fixltx2e}
\usepackage{graphicx}
\usepackage{multicol}
\usepackage[normalem]{ulem}
%% Please revise the following command, if your babel
%% package does not support es-ES
\usepackage{color}
\usepackage{hyperref}

\pagestyle{empty}
%\makelabels
\topmargin -1.0 cm
\textheight 20.5 cm
\geometry{top=2cm, left=3cm, right=3cm, bottom=3cm}
 
\begin{document}


\begin{letter}
{}

\fbox{
    \parbox{15cm}{
        \setlength{\baselineskip}{18pt}
        \textbf{\textcolor[rgb]{0.000,0.502,0.502}{{\LARGE Acta de reunión de 
                Dr.Scratch:}}}\\
        \setlength{\baselineskip}{0pt}
        \textbf{\textcolor[rgb]{0.000,0.502,0.502}{{\LARGE Fomentando la 
                    creatividad y vocaciones científicas con Scratch}}}\\    
    }
}\\

\vspace{1cm}
\setlength{\parskip}{0.5mm}
\parbox{70mm}{
    \textbf{
        \textcolor[rgb]{0.000,0.502,0.502}{\large Asistentes:}
        \begin{itemize}
            \setlength{\parskip}{0.5mm}
            \item{\textcolor[rgb]{0.502,0.502,0.502}{Gregorio Robles}}
            \item{\textcolor[rgb]{0.502,0.502,0.502}{Jesús Moreno}}
            \item{\textcolor[rgb]{0.502,0.502,0.502}{Eva Hu}}
            \item{\textcolor[rgb]{0.502,0.502,0.502}{Mari Luz Aguado}}
        \end{itemize}
    }
}
\hfill
\parbox{90mm}{
    \bf{
        \textcolor[rgb]{0.000,0.502,0.502}{\large Entidad:}
        \textcolor[rgb]{0.502,0.502,0.502}{Universidad Rey Juan Carlos}\\ 
        \textcolor[rgb]{0.000,0.502,0.502}{\large Fecha:}
        \textcolor[rgb]{0.502,0.502,0.502}{3 de junio de 2015} \\
        \textcolor[rgb]{0.000,0.502,0.502}{\large Lugar:}\\
        \textcolor[rgb]{0.502,0.502,0.502}{Camino del Molino s/n \\
                                           28943 Fuenlabrada (Madrid)} \\
      
  
   }
}



\vspace{0.75cm}
\textbf{{\LARGE Temas tratados}}
\vspace{0.5cm}


En la reunión realizada el \textbf {22 de julio de 2015} en el {\bf Campus de
Fuenlabrada de la Universidad Rey Juan Carlos} y dirigida por Dr.Gregorio Robles
se trataron los siguientes temas:\\

\begin{enumerate}
    
    \item{\textbf {Pantallas nuevas:}}
    \begin{itemize}
            \item {Se han añadido nuevas funcionalidades a las pantallas de análisis:}
                \begin{itemize}
                    \item {BootstrapTour para ir guiando al usuario a través de la pantalla.}
                     \item {Los malos hábitos de programación se muestran ahora como un cuadro emergente cuando se pasa el ratón sobre cada uno de ellos.}
                     \item {Se ha añadido un dibujo diferente a cada nivel.}
                 \end{itemize}
            \item {En las pantallas que explican cada uno de las habilidades de programación se han cambiado las imágenes por código de Scratchblocks.}
            \item {Se han corregido algunos errores.}
    \end{itemize}
    
    \item{\textbf {Organizaciones:}}
    \begin{itemize}
            \item {Se ha creado la funcionalidad que resetea la contraseña.}
            \item {Se ha hablado de cambiar el formato en el que se entrega el CSV.}
            \item {Se comenta que se podría solicitar un ID junto con la url del proyecto.}
    \end{itemize}

    \item{\textbf {Scratch Conference:}}
    \begin{itemize}
        \item {Se ha repartido la información a comentar en las distintas charlas que impartiremos en la conferencia.}
    \end{itemize}

\end{enumerate}

\vspace{2cm}
\textbf{{\LARGE Revisión de números alcanzados}}
\vspace{0.5cm}

\begin{enumerate}  
    \item {\textbf{Plataforma de Dr.Scratch:}}
        \begin{itemize}
            \item {Número de proyectos analizados: 7764 proyectos.}
            \item {Número de lenguas de la plataforma: 2 (inglés y castellano).}
            \item {Número de países que visitan la plataforma: 10 países.}
            \item {Número de sesiones: 1336.}
            \item {Número de usuarios: 1184.}
            \item {Número de páginas vistas: 3390.}
        \end{itemize}
    \item {\textbf{Twitter:}}
        \begin{itemize}
            \item {Número de tweets: 4.}
            \item {Visitas al perfil:464.}
            \item {Menciones:31.}
            \item {Nuevos seguidores:42.}
        \end{itemize}
    \item {\textbf{Blog:}}
        \begin{itemize}
            \item {Visitas: 313.}
            \item {Visitantes: 189.}
            \item {Entradas publicadas: 2.}
        \end{itemize}

\end{enumerate}

\end{letter}

\end{document}
